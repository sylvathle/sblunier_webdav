\subsection{The jobs}

Until now we saw how to use Davix and how we want to analyse the events rates. But to do that it's also importante to understand how to send the jobs and to do that it's necessary to understand the command \textbf{prun}.

The more important to know about prun is how to use... the manual ! All the informations in this part are in th manual.

\lstset{language=bash}
\begin{lstlisting}
	prun --help
\end{lstlisting}

A CERN-twiki is also available as a tutorial where you can find more informations about what is prun:\\
\begin{center}
	\url{https://twiki.cern.ch/twiki/bin/view/PanDA/PandaRun}
\end{center}

In some words: \textit{"prun is a Panda-client software which allows users to submit general jobs to Panda"} (source: \url{https://twiki.cern.ch/twiki/bin/view/PanDA/PandaRun}). Here we will see how we used prun for our specific davix-jobs.\\

We will show an complete example, for that we asume we have a simple program \textit{main.C} that opens a file and read is as presented before but without any measurement. 

\begin{lstlisting}
prun --outDS user.sblunier.ariac001.ANALY_DESY-ZN --rootVer=5.34.19-davix_p1 --cmtConfig=x86_64-slc6-gcc48-opt --bexec make --exec "./main https://voatlasrucio-redirect-prod-01.cern.ch/redirect/mc12_8TeV/NTUP_COMMON.01272994._000002.root.1?rse=DESY-ZN_DATADISK nocache" --site=ANALY_DESY-ZN
\end{lstlisting}

The arguments means:

\begin{itemize}
	\item \textit{--outDS} user.sblunier.ariac001.ANALY\_DESY-ZN: name of the folder where we will be created the outputs of the jobs, it must begin by user.\textit{username} replacing \textit{username} by the correct name.
	\item \textit{--rootVer}=5.34.19-davix\_p1 design the root version you will use to run the job
	\item \textit{--cmtConfig}=x86\_64-slc6-gcc48-opt to design the correct compiled version
	\item \textit{--bexec} make: what will be run before the execution, here we just want to compile our program \textit{main.C}
	\item \textit{--exec} "./main https://voatlasrucio-redirect-prod-01.cern.ch/redirect/mc12\_8TeV/NTUP\_COMMON.01272994.\_000002.root.1?rse=DESY-ZN\_DATADISK nocache": command that will run the job, here \textit{main} with two arguments.
	\item \textit{--site}=ANALY\_DESY-ZN: To ask the job to be ran where we want
\end{itemize}

Note that this command could be much simpler, many of those arguments have default values but here we need to fix those parameters.\\

There are other parameters that will be used in the monitoring (see section \ref{sec:monitoring}):

\begin{itemize}
	\item \textit{--extFile}=proxy: We can send an external file with the job, here we need the proxy to allow remote access from the processing site.
	\item \textit{--outputs}=rootfile.root: File generated by the job that we will need to recover.
\end{itemize}
