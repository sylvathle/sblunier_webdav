
\subsection{Davix integration in ROOT}

The integration of Davix in ROOT was done on lxplus. Explanation are given to built root for cvmfs, the ROOT and Davix folders will be separated.\\

\subsubsection{Get Root}

In a folder where you will want to build your project, get the right version of ROOT:

\begin{lstlisting}
	> git clone http://root.cern.ch/git/root.git
\end{lstlisting}

To select the needed version, it is useful to know which one are available with:

\begin{lstlisting}
	> cd root
	> git tag -l
\end{lstlisting}

then to get the right version:

\begin{lstlisting}
	> git checkout v5-34-19
\end{lstlisting}

replacing "v5-34-19" by the version you want.


%%%%%%%%%%%%%%%%%%%%%%%%%%%%%%%%%%%%%%%%%%%%%%%%%%%%%%%%%%%%%%%%%%%%%%%%%%%%%%%%%%%%
\subsubsection{Get and compile Davix}

Out of the recent downloaded ROOT folder download git from the public repository (here we will give the commands assuming that davix/ and root/ are in the same directory):

\begin{lstlisting}
	> (cd ..)
	> git clone http://git.cern.ch/pub/davix
\end{lstlisting}

Just like we did for look at the version you want with:


\begin{lstlisting}
	> cd davix
	> git tag -l
\end{lstlisting}

and select the right version:


\begin{lstlisting}
	> git checkout R_0_3_4
\end{lstlisting}

Now we will build Davix, to do that you can follow the steps here or you can also look at \textit{README} where there are different ways to compile Davix, here we detail our way to compile Davix in embedded mode.


\begin{lstlisting}
	> mkdir -p build
	> git submodule update --recursive --init
	> cd build
	> cmake -D BOOST_EXTERNAL=NO -D CMAKE_INSTALL_PREFIX=/ ..
	> make
\end{lstlisting}

Now we will get the libraries to make possible integration in Root (still in the \textit{build} folder):


\begin{lstlisting}
	> mkdir -p root
	> make install DESTDIR=root/
\end{lstlisting}

Now you should have the libraries in davix/build/root/lib64/

%%%%%%%%%%%%%%%%%%%%%%%%%%%%%%%%%%%%%%%%%%%%%%%%%%%%%%%%%%%%%%%%%%%%%%%%%%%%%%%%%%
\subsubsection{Build ROOT}

The first step is to configure ROOT, then go first in your root folder and configure it with:

\begin{lstlisting}
	> (cd ../../../root)
	> ./configure --with-xrootd=/afs/cern.ch/sw/lcg/external/xrootd/3.2.7/x86_64-slc6-gcc48-opt --enable-davix --with-davix-incdir=../davix/include/davix/ --with-davix-libdir=../davix/build/root/lib64/
\end{lstlisting}

the first argument of \textit{configure} can be set only on cvmfs, if you only want root+davix without xrootd replace it by \textit{--disable-xrootd}. Make sure davix and xroot appear in the list of \textit{Enable support} that is printed at the end of the output of \textit{./configure}, obviously if you don't want to integrate xrootd it should be in the list.\\

Now compile root:

\begin{lstlisting}
	> make
\end{lstlisting}

Uncomment the two following lines in the file etc/system.rootrc:

\begin{lstlisting}
	709 Davix.GSI.CACheck: y
	717 Davix.GSI.GridMode: y
\end{lstlisting}

and finally, \underline{only if you don't want this version to be propagated on cvmfs} copy the compiled libraries of Davix in the lib folder of root:

\begin{lstlisting}
	> cd ..
	> cp davix/build/root/lib64/* root/lib/
\end{lstlisting}

Select the recent build patch with:

\begin{lstlisting}
	> source root/bin/thisroot.sh
\end{lstlisting}

%%%%%%%%%%%%%%%%%%%%%%%%%%%%%%%%%%%%%%%%%%%%%%%%%%%%%%%%%%%%%%%%%%%%%%%%%%%%%%%%%%%%

\subsubsection{Direct building from ROOT}

It is possible not to do the second part. Run the script \textit{installDavix.sh}:

\begin{lstlisting}
	> cd root/
	> ./build/installDavix.sh
\end{lstlisting}

It will download direclty Davix but you will have no choice on its version. You can also to that with xrootd in fact with \textit{installXrootd.sh}. The compilation of Davix will also have to be done separatly.
