\section{Softwares that are involved}

\begin{itemize}
\item Root : C++ Framework developped by cern, to analyze large amount of datas but was extended to many other funcionalities like. It has an interpreted command language it means that you are not obliged to compile it to make it work. It has been thinking to make possible parallel queries on databases that can be on clusters or many-core machines. \\

\item Rucio : New version of ATLAS Distributed Data Management, it was developped to manage the large amount of datas from the dector. It manages accounts, datasets and distributed storage systems.\\ (source: \url{http://rucio.cern.ch/}) 

\item Davix : Davix is a lightweight library and a set of command line tools for file access and file management with HTTP Based protocols. Davix aims to be an easy-to-use, reliable and performant I/O layer for Cloud and Grid Storages.\\ (source: \url{http://dmc.web.cern.ch/projects/davix/home})

\item Athena\\

\item http : HTTP stands for Hypertext Transfer Protocol. It's the network protocol used to deliver virtually all files and other data (collectively called resources) on the World Wide Web, whether they're HTML files, image files, query results, or anything else. (source: \url{http://www.jmarshall.com/easy/http/whatis})
\end{itemize}
